%MIT License

%Copyright (c) 2018 Orange Lee

%Permission is hereby granted, free of charge, to any person obtaining a copy
%of this software and associated documentation files (the "Software"), to deal
%in the Software without restriction, including without limitation the rights
%to use, copy, modify, merge, publish, distribute, sublicense, and/or sell
%copies of the Software, and to permit persons to whom the Software is
%furnished to do so, subject to the following conditions:

%The above copyright notice and this permission notice shall be included in all
%copies or substantial portions of the Software.

%THE SOFTWARE IS PROVIDED "AS IS", WITHOUT WARRANTY OF ANY KIND, EXPRESS OR
%IMPLIED, INCLUDING BUT NOT LIMITED TO THE WARRANTIES OF MERCHANTABILITY,
%FITNESS FOR A PARTICULAR PURPOSE AND NONINFRINGEMENT. IN NO EVENT SHALL THE
%AUTHORS OR COPYRIGHT HOLDERS BE LIABLE FOR ANY CLAIM, DAMAGES OR OTHER
%LIABILITY, WHETHER IN AN ACTION OF CONTRACT, TORT OR OTHERWISE, ARISING FROM,
%OUT OF OR IN CONNECTION WITH THE SOFTWARE OR THE USE OR OTHER DEALINGS IN THE
%SOFTWARE.

\documentclass[UTF8]{article}
\usepackage{ctex}
\usepackage{hyperref}	% 用于交叉引用
\usepackage{setspace}	% 用于设置行间距
\usepackage{listings}	% 用于代码高亮
\usepackage{xcolor}		% 用于处理颜色
\usepackage{ulem}		% 用于各种线
\usepackage{amsmath}	% 用于数学公式(如 \begin{align})
\usepackage{amsthm}		% 用于数学版式(如 \newtheorem{cmd}{caption})
\usepackage{amsfonts}
\usepackage{amssymb}
\usepackage{booktabs}	% 用于表格画线
\usepackage{graphicx}	% 用于插入图片
\usepackage[top = 0.8in, bottom = 0.8in, left = 0.8in, right = 0.8in]{geometry} %设置页边距

% 设置页脚页脚
\usepackage{fancyhdr}
\usepackage{lastpage}	% 用于获取总页数
\pagestyle{fancy}
\fancyhf{}
\cfoot{第 \thepage 页,共 \pageref{LastPage} 页}
\chead{\insertsubject}
%\renewcommand \headrulewidth {0pt} % 删除页眉横线

% 设置代码
\newfontfamily \Consolas {Consolas}
\definecolor {CPPColor} {HTML} {2B91AF}
\lstset{
	columns = fixed,
	numbers = left,                                        % 在左侧显示行号
	frame = none,                                          % 不显示背景边框
	backgroundcolor = \color[RGB]{245,245,245},            % 设定背景颜色
	keywordstyle = \color[RGB]{40,40,255},                 % 设定关键字颜色
	numberstyle = \footnotesize\color{darkgray},           % 设定行号格式
	basicstyle = \Consolas,                                % 设定代码字体
	commentstyle = \yahei\color[RGB]{0,96,96},             % 设置代码注释的格式
	stringstyle = \yahei\slshape\color[RGB]{128,0,0},      % 设置字符串格式
	showstringspaces = false,                              % 不显示字符串中的空格
	language = c++,                                        % 设置语言
	morekeywords={alignas,continute,friend,register,true,alignof,decltype,goto,
	reinterpret_cast,try,asm,defult,if,return,typedef,auto,delete,inline,short,
	typeid,bool,do,int,signed,typename,break,double,long,sizeof,union,case,
	dynamic_cast,mutable,static,unsigned,catch,else,namespace,static_assert,using,
	char,enum,new,static_cast,virtual,char16_t,char32_t,explict,noexcept,struct,
	void,export,nullptr,switch,volatile,class,extern,operator,template,wchar_t,
	const,false,private,this,while,constexpr,float,protected,thread_local,
	const_cast,for,public,throw,std},
	emph={map,set,multimap,multiset,unordered_map,unordered_set,
	unordered_multiset,unordered_multimap,vector,string,list,deque,
	array,stack,forwared_list,iostream,memory,shared_ptr,unique_ptr,
	random,bitset,ostream,istream,cout,cin,endl,move,default_random_engine,
	uniform_int_distribution,iterator,algorithm,functional,bing,numeric,
	T}, % 自定义
	emphstyle=\color{CPPColor},
}
\newcounter {icode}
\newcommand \inserticode {\arabic{icode}}
\newcommand \illuscode[1]
{
	\stepcounter{icode}
	\begin{center}
		\textbf{代码 \inserticode:} #1
	\end{center}
}
\usepackage{color}
\newcommand \code[1] {\colorbox[RGB]{245,245,245}{\Consolas #1}}

\newcommand \insertsubject {{Chapter 1 树状数组}}

\hypersetup
{
	pdfauthor = Orange Lee,
	pdftitle = \insertsubject,
	pdfsubject = \insertsubject,
	pdfkeywords = \insertsubject,
	colorlinks = true,
	linkcolor = black,
	anchorcolor = black,
	citecolor = black,
	urlcolor = black
}

\title{\insertsubject}
\author{}
\date{}

\begin{document}

	\maketitle
	\tableofcontents
	\thispagestyle{empty}

	\clearpage
	\setcounter{page}{1}

	\section{什么是树状数组}

	树状数组,又叫二叉索引树(Binary Indexed Tree,BIT),由 Peter Fenwick 于 1994 年发明(故又称 Fenwick 树),旨在维护序列的前缀和。\footnote{Wikipedia}
	它的代码极为简单,广为人知:
	\begin{lstlisting}
template <typename T, int size = 0xffff>
class BIT
{
	static inline int lowbit(const int& x)
	{
		return x & (-x);
	}
	T b[size + 1];

public:
	void add(int pos, const T& val)
	{
		while (pos <= size)
		{
			b[pos] += val;
			pos += lowbit(pos);
		}
	}
	T prefix(int pos)
	{
		T ret{};
		while (pos)
		{
			ret += b[pos];
			pos ^= lowbit(pos);
		}
		return ret;
	}
};
	\end{lstlisting}
	\illuscode{常规的 BIT}

	为了方便,我们称代码 1 中的 $b$ 数组为 \textit{BIT 数组}或 \textit{$b$ 数组}。

	代码 1 中的类型 \code{T} 需要且仅需要 \code{+=} 运算符,因此整数和浮点数都能作为 BIT 维护的类型。若类型 \code{T} 为 \code{set}\footnote{这里的 \code{set} 指抽象的集合。},且把 \code{set += set} 看作两个 \code{set} 的合并,那么 \code{set} 也能作为一个合法的 \code{T} 类型。

	在代码仓库中另附有更简洁常用的代码。需要注意的是,$b$ 数组的下标只能从 \code{1} 开始,因为 \code{add(0)} 的调用将会陷入死循环。但 \code{prefix(0)} 的调用却不会陷入死循环,因此我们在 \code{prefix} 函数中不用进行特判。

	\section{常规的 BIT}

	\newcommand \lowbit {\mathrm{lowbit}}

	\subsection{$\lowbit$ 运算}

	顾名思义,$\lowbit(x)$ 表示在二进制下,仅保留 $x$ 的最低非零位后得到的数。例如,$\lowbit((01101010)_2) = (00000010)_2$。

	计算 $\lowbit(x)$ 的思路主要有两种。一是逐位检查;二是将原数减去一,相当于将 $\lowbit(x)$ 的对应位以及更低位全部取反。然后与原数进行按位与运算,用原数减去按位与的结果便是 $\lowbit(x)$。

	另外,可以验证,若我们用补码表示二进制数,那么有 \code{lowbit(x) = x \& -x}。这种方法实质上就是前面的第二种思路,只不过形式上更加简洁。

	\subsection{BIT 的原始定义}

	设原数组为 $a_i \pod {1 \le i \le n}$,其前缀和数组为 $s$。规定 $b_i \pod {1 \le i \le n}$ 满足:
	$$
	b_x = \sum_{i = \left(x \oplus \lowbit(x)\right) + 1}^{x} a_i
	$$

	\bigskip

	以上公式比较抽象。换个形式来看,$b_x$ 求的是 $a$ 从 $\left( (A 0 \overbrace{\cdots}^{\text{全是 }0} 01)_2 \right)$ 到 $\left( x = (A 1 \overbrace{\cdots}^{\text{全是 }0} 00)_2 \right)$ 的和,其中 $A$ 代表任意 $01$ 串。另外,也可用 $s$ 表示 $b$:
	$$
	b_x = s_x - s_{x \oplus \lowbit(x)}
	$$

	例如,对 $x = (1010)_2$,有:
	$$
	b_{(1100)_2} = a_{(1001)_2} + a_{(1010)_2} +
	a_{(1011)_2} + a_{(1100)_2}
	$$

	\subsection{正确性的证明}

	\newcommand \prefix {\mathrm{prefix}}
	\paragraph{证明 \code{prefix} 函数的正确性}

	分析:
	观察 \code{prefix} 函数,它等价于:
	$$
	\prefix(x) =
	\begin{cases}
		b_x & x = \lowbit(x)
		\\
		s_{x \oplus \lowbit(x)} + b_x & \text{OTHERWISE}
	\end{cases}
	$$

	我们要证:$s_x = \prefix(x)$。

	\begin{proof}

		分两种情况讨论。

		(1)当 $x = \lowbit(x)$ 时,$s_x = b_x$,即 $s_x = \prefix(x)$。

		(2)当 $x \ne \lowbit(x)$ 时,设 $y = x \oplus \lowbit(x)$。有:
		\begin{align*}
			s_x &= s_y + (s_x - s_y)
			\\&=
			s_y + b_x
		\end{align*}

	\end{proof}

	\paragraph{证明 \code{add} 函数的正确性}

	分析:
	我们首先考虑令 $a_x$ 加上了 $v$ 后,哪些 $b_x$ 也需要加上 $v$。

	以 $x = (00101010)_2$ 为例,模拟代码运行,发现需要考虑的 $b$ 有:
	\begin{align*}
		b_{(00101010)_2}
		\\
		b_{(00101100)_2}
		\\
		b_{(00110000)_2}
		\\
		b_{(01000000)_2}
		\\
		b_{(10000000)_2}
	\end{align*}

	可以发现,以上对应下标是这样得来的:考虑 $x$ 中所有比 $\lowbit$ 位高的零位。对于每一个这样的零位,将低于它的位全部置零,将它本身置 $1$,保持高于它的位不变,就得到了一个下标。特别地,$x$ 本身也是一个下标。我们要证明的是,对于所有这些下标,都应加上 $v$;对于不属于此列的下标,都不应加上 $v$。

	\begin{proof}

		设下标为 $y$。

		(1)当 $y = x$ 时,根据 $b_y$ 的定义,一定需要加上 $v$。

		(2)当 $y \ne x$ 时,若 $b_y$ 需要加上 $v$,则要满足:
		\begin{equation*}
			y \oplus \lowbit(y) < x \le y \tag{1}
		\end{equation*}

		当 $y \ne x$ 时,设 $y$ 和 $x$ 的形式分别为:
		\begin{align*}
			y = \overline{A10 \overbrace{\cdots}^{\text{全是 0}}}
			\\
			x = \overline{B?? \overbrace{\cdots}^{\text{全是 ?}}}
		\end{align*}

		\leftline{充分性:}

		在用 \code{add} 的方法构造 $y$ 时,$y$ 从高到低第一个与 $x$ 不同的位一定为 $1$,故满足:$x < y$。

		而将 $y$ 第一个与 $x$ 不同的位置为 $0$ 后,由于在该位之后 $x$ 的位一定不都为 $0$(至少还有 $\lowbit(x)$),故满足:$y \oplus \lowbit(y) < x$。

		\bigskip

		\leftline{必要性:}

		只看 $A$ 和 $B$,必须满足 $A = B$。若 $A < B$,那么不满足 $x < y$;若 $A > B$,那么不满足 $y \oplus \lowbit(y) < x$。当 $A = B$ 时,我们可以知道 $B$ 后第一位一定是 $0$,且后面一定有非 $0$ 位,才能满足 $y + \lowbit(y) < x$。而这种情况正好对应(2)中讨论的情况。故不存在其它下标满足 $(1)$ 式。

	\end{proof}

	\subsection{时间复杂度}

	我们一般认为 BIT 查询与修改的时间复杂度均为 $O(\log n)$。还能得到更为准确的循环次数。对于查询操作,由于每次循环会使非零位的个数减一,故循环次数为 $\mathrm{popcount}(x)$;\footnote{$\mathrm{popcount}$ 表示二进制数中非零位的个数。}对于修改操作,操作次数为 $\left\lceil \log_2 n \right\rceil - \mathrm{popcount}(x) + 1$。

	\subsection{更好地理解 BIT}

	以上概念和证明略显抽象,下面是一个很自然的理解方式。\footnote{\url{https://blog.csdn.net/flushhip/article/details/79165701}}

	\section{BIT 的应用}

	\subsection{维护前缀信息}

	\subsubsection{维护前缀和}

	BIT 最基本的用途就是维护一个序列的前缀和,它能在 $O(\log n)$ 的时间复杂度内计算出 $s_k = \sum_{i = 1}^{k} a_i \pod {k \le n, k \in \mathbb{N}}$。由于区间和可以用前缀和相减的形式表示:
	$$
	\sum_{i = l}^{r} a_i = s_r - s_{l - 1}
	$$
	因此 BIT 可以在 $O(\log n)$ 的时间复杂度内单点修改,在 $O(2 \log n)$ 的时间复杂度内查询区间和。

	\paragraph{初始化维护前缀和的 BIT 的方法}

	假设有一个 BIT,对应的数组为 $b_{1 \sim n}$,一开始 $b_i = 0$。现在给定一个下标范围为 $1 \sim n$ 的 $a$ 数组,希望用这个 BIT 维护 $a$。如何初始化 $b$ 数组?

	一个显然的方法是调用 $n$ 次 \code{add} 函数,时间复杂度为 $O(n \log n)$。

\end{document}